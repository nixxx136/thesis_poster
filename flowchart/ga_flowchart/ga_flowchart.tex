%%%%%%%%%%%%%%%%%%%%%%%%%%%%%%%%%%%%%%%%%%%%%%%%%%%%%%%%%%%%%%%
%
% Welcome to Overleaf --- just edit your LaTeX on the left,
% and we'll compile it for you on the right. If you give
% someone the link to this page, they can edit at the same
% time. See the help menu above for more info. Enjoy!
%
% Note: you can export the pdf to see the result at full
% resolution.
%
%%%%%%%%%%%%%%%%%%%%%%%%%%%%%%%%%%%%%%%%%%%%%%%%%%%%%%%%%%%%%%%
\documentclass{article}

\usepackage[latin1]{inputenc}
\usepackage{tikz}
\usetikzlibrary{positioning,shapes,arrows}

%%%<
\usepackage{verbatim}
\usepackage[active,tightpage]{preview}
\PreviewEnvironment{tikzpicture}
\setlength\PreviewBorder{5pt}%
%%%>

\begin{comment}
:Title: Simple flow chart
:Tags: Diagrams

With PGF/TikZ you can draw flow charts with relative ease. This flow chart from [1]_
outlines an algorithm for identifying the parameters of an autonomous underwater vehicle model. 

Note that relative node
placement has been used to avoid placing nodes explicitly. This feature was
introduced in PGF/TikZ >= 1.09.

.. [1] Bossley, K.; Brown, M. & Harris, C. Neurofuzzy identification of an autonomous underwater vehicle `International Journal of Systems Science`, 1999, 30, 901-913 


\end{comment}

\begin{document}
\pagestyle{empty}

% Define block styles
\tikzstyle{decision} = [diamond, draw, fill=blue!20, 
    text width=4.5em, text badly centered, node distance=2.5cm, inner sep=0pt]
\tikzstyle{block} = [rectangle, draw, fill=blue!20, node distance=2cm,
    text width=6em, text centered, rounded corners, minimum height=1em]
\tikzstyle{line} = [draw, -latex']
\tikzstyle{cloud} = [draw, ellipse,fill=red!20, node distance=3cm,
    minimum height=2em]
    
\begin{tikzpicture}[node distance = 2cm, auto]
    % Place nodes
    \node [block] (init) {start};
%    \node [cloud, left of=init] (expert) {expert};
%    \node [cloud, right of=init] (system) {system};
    \node [block, text width=10em, below =1cm of init] (initpop) {generate initial population};
    \node [block, below =1cm of initpop] (calcfitness) {calculate fitness of each individual };
%     \node [block, below of=initpop] (evaluate) {evaluate candidate models};
%     \node [block, left of=evaluate, node distance=3cm] (update) {update model};
    \node [block, right of = calcfitness, right=1cm, yshift=-.5cm] (selection) {selection};
    \node [block, below of = selection, yshift=.5cm] (crossover) {crossover};
    \node [block, below of = crossover, yshift=.5cm] (mutation) {mutation};
    \node [block, below of=calcfitness] (newpop) {create new population};
    \node [decision, below of=newpop] (decide) {end condition};
    \node [block, below of=decide, node distance=2cm] (stop) {stop};
    % Draw edges
    \path [line] (init) -- (initpop);
    \path [line] (initpop) -- (calcfitness);
    \path [line] (calcfitness) -- (newpop);


    \path [line] ([yshift=0.1cm] newpop.east) -| ([xshift=-.82cm] selection.west)
                          -- (selection.west);
    \path [line] (selection) -- (crossover);
    \path [line] (crossover) -- (mutation);
    \path [line] (mutation) -| ([xshift=1cm, yshift=-0.1cm] newpop.east)
                            -- ([yshift=-0.1cm] newpop.east);
    
    \path [line] (newpop) -- (decide);
    \path [line] (decide) -| node [near start] {no} ([xshift=-1cm] newpop.west)
%                         |- ([yshift=+1cm, xshift=-.2cm] init.north) coordinate (aux)
                        -- (newpop);
%     \path [line] (decide) |- node[near start] ([xshift=-1cm] newpop.west) {yes} (newpop);
%     \path [line] (update) |- (init-pop);
    \path [line] (decide) -- node {yes}(stop);
\end{tikzpicture}


\end{document}